%\section{Motivación}
\section{Introducción}

En los últimos años, la educación en línea ha cobrado una gran relevancia debido a la pandemia global que obligó a muchos países a implementar medidas de distanciamiento social. Como resultado, muchas instituciones educativas han tenido que adaptarse rápidamente a los sistemas de aprendizaje en línea. Entre estos sistemas, los Learning Management System (LMS) son muy populares debido a su capacidad para ofrecer una plataforma en línea que permite a los profesores compartir recursos y comunicarse con sus alumnos.

En este contexto, \verb|GitHub Education| \cite{github-education} ha surgido como una plataforma líder en la enseñanza de programación y desarrollo de proyectos en equipo. Para ello, se utiliza \verb|GitHub Classroom| \cite{github-classroom}, una herramienta que permite a los profesores crear tareas y asignaciones de manera automatizada y que se integra perfectamente con GitHub. Sin embargo, una limitación de GitHub Classroom es que solo permite crear tareas y asignaciones que sean entregadas como repositorios de código en GitHub.

Con el objetivo de abordar esta limitación, se propone una aplicación que utiliza la filosofía de GitHub Classroom pero que permite la creación de páginas estáticas para LMS. La aplicación utiliza \verb|GitHub GraphQL| \cite{github-graphql} para obtener información de los repositorios y \verb|Astro| \cite{astro} para generar las páginas estáticas.

\section{Antecedentes y estado actual del tema}

Existen algunas herramientas en línea populares para la creación de páginas web para la enseñanza, como \verb|Moodle| \cite{moodle} y \verb|Google Classroom| \cite{google-classroom}. Sin embargo, estas herramientas están diseñadas para la educación en general y no específicamente para la enseñanza de programación. Además, aunque son gratuitas, pueden requerir una curva de aprendizaje para su uso efectivo.

Por otro lado, \verb|GitLab| \cite{gitlab} y \verb|Bitbucket| \cite{bitbucket} son plataformas populares para la colaboración y gestión de proyectos de programación. Aunque ambos permiten la creación de páginas web para repositorios, su enfoque principal no es la creación de un LMS completo. GitHub Classroom, por otro lado, es una extensión de GitHub que se centra específicamente en la enseñanza de programación. Permite a los profesores crear repositorios y asignaciones para los estudiantes, y ofrece una integración más efectiva con GitHub GraphQL o \verb|GitHub API REST| \cite{github-rest}.

La aplicación propuesta utiliza GitHub Classroom, GitHub GraphQL y Astro para proporcionar una alternativa viable y accesible para la creación de páginas web para LMS de programación en línea.

Astro es una herramienta que permite la creación de páginas web estáticas utilizando componentes de JavaScript. Es fácil de usar y tiene una sintaxis clara y concisa, lo que la hace ideal para la creación de páginas web para LMS. Aunque existen otras herramientas para generación de web estáticas como \verb|Jekyll| \cite{jekyll} o \verb|Hugo| \cite{hugo}, la cuales han sido muy populares en los últimos años, en este caso se ha decidido por usar Astro, un framework nuevo y novedoso.

Esto se debe a que algunas de las principales desventajas de Jekyll y Hugo es que pueden tener una curva de aprendizaje pronunciada para los nuevos usuarios. La sintaxis y la configuración pueden ser complejas para aquellos que no tienen experiencia previa en la creación de sitios web estáticos. Además, a medida que un proyecto se vuelve más complejo, Jekyll y Hugo pueden requerir más tiempo y esfuerzo para mantener y actualizar.

Mientras que las razones por la que se han elegido a Astro han sido su facilidad de uso, su sintaxis clara y concisa, la oportunidad que ofrecer al integrarse con otros frameworks para UI, como \verb|React| \cite{react} o Angular\cite{angular}, su integración directa con la metodología \verb|Jamstack| \cite{jamstack}, su fácil personalización a través de plugins, etc.

Además de GitHub GraphQL, GitHub también proporciona una API REST que permite a los desarrolladores interactuar con los repositorios de GitHub. Aunque la API REST no proporciona la misma cantidad de información detallada que GraphQL, sigue siendo una herramienta poderosa para la integración con GitHub.

En comparación con GraphQL, la API REST puede ser más fácil de usar para proyectos más pequeños y simples. También es posible que algunos desarrolladores prefieran la sintaxis y el enfoque de REST sobre GraphQL. Sin embargo, para proyectos más complejos que requieren más información detallada, GraphQL puede ser una mejor opción.
