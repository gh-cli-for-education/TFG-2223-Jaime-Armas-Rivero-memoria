``Software Engineering Is Programming Integrated Over Time'' \label{conclusion}

\bigskip
Lo que más resalto en este trabajo de fin de grado ha sido la tarea de diseñar un ecosistema modulable desde cero con unas metas marcadas y con el ideal de crear un software que no necesite de cambios en su estructura para crecer, que sea simple e intuitivo, pero útil y que permita la colaboración. Conseguir todas estas características ha requerido una amplia etapa de diseño y prototipado donde las ideas se implementaban y descartaban por igual.\\
También otro apartado que consumió bastante tiempo fue el manejo de errores. Al ser una aplicación que realiza muchas llamadas, lee y escribe de ficheros y le pide \emph{input} al usuario, hay muchos puntos donde la aplicación puede fallar. Manejar estos errores y pulir la aplicación en general es trabajo que no genera un resultado inmediato, pero hace que la aplicación sea más mantenible y estable.\\
Por otro lado, me siento muy afortunado de haber podido trabajar con tecnologías con las que me siento cómodo como JavaScript, Go y \verb|fzf|, pero han sido más las que he aprendido y dada la buena experiencia que he tenido con: \verb|jq|, \verb|gh cli| y \verb|GraphQL| comentar que seguramente las utilice en futuros proyectos.\\
No obstante, me apena no haber podido cumplir con el objetivo de interoperabilidad entre comandos. El comando \verb|view| se podría beneficiar enormemente si pudiese leer de la \emph{standard input} para recibir información de un comando como \verb|data|, pero preferí enfocar el tiempo en mejorar la calidad y usabilidad de todo el ecosistema.\\
El trabajo de fin de grado está terminado, pero el proyecto puede crecer tanto como desee. Este tipo de proyecto requiere de trabajo constante y prolongado para que tenga acogida en la comunidad. Mis intenciones son seguir contribuyendo a este proyecto y tomarlo como mi proyecto personal y mi vía de contribución al open source.