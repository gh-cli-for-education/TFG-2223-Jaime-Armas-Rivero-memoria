\newglossaryentry{API}
{
    name=API,
    description={Una \textbf{interfaz de programación de aplicaciones} es una manera de que dos o más programas informáticos se comuniquen entre sí. Es un tipo de interfaz de software que ofrece un servicio a otras piezas de software}
}

\newglossaryentry{CLI}
{
    name=CLI,
    description={Es un tipo de interfaz de usuario de computadora que permite a los usuarios dar instrucciones a algún programa informático o al sistema operativo por medio de una línea de texto simple}
}

\newglossaryentry{CWD}
{
    name=CWD,
    description={CWD (Current Working Directory) se refiere al directorio actual en el que se está trabajando dentro de un sistema de archivos. Es la ubicación o carpeta en la que un programa o usuario se encuentra actualmente realizando operaciones o ejecutando comandos}
}

\newglossaryentry{JSX}
{
    name=JSX,
    description={Es una extensión de JavaScript que combina HTML y JavaScript para facilitar la creación de interfaces de usuario en aplicaciones web, especialmente en frameworks como React.}
}

\newglossaryentry{framework}
{
    name=framework,
    description={Un framework es una estructura de software que proporciona herramientas y bibliotecas para facilitar el desarrollo de aplicaciones. Actúa como una base sobre la cual se construyen aplicaciones, ahorrando tiempo al ofrecer soluciones predefinidas para tareas comunes. Los frameworks permiten a los desarrolladores enfocarse en la lógica de su aplicación en lugar de preocuparse por detalles de implementación, lo que agiliza el proceso de desarrollo y promueve la reutilización de código.}
}

\newglossaryentry{MPA}
{
    name=MPA,
    description={Multi-Page Application (MPA) se refiere a un tipo de aplicación web que consta de varias páginas individuales. Cada página representa una vista diferente de la aplicación y se carga por separado del servidor cuando el usuario navega entre ellas. A diferencia de las Single-Page Applications (SPA), que cargan una única página y actualizan su contenido de forma dinámica, las MPAs requieren una carga completa de cada página, lo que puede generar una experiencia más tradicional de navegación.}
}

\newglossaryentry{SSR}
{
    name=SSR,
    description={Server-Side Rendering (SSR) es un enfoque en el desarrollo web donde el servidor genera y envía la página completa al cliente como HTML listo para ser mostrado, ofreciendo una carga inicial más rápida.}
}

\newglossaryentry{SSG}
{
    name=SSG,
    description={SSG (Static Site Generation) es un enfoque de desarrollo web donde las páginas web se generan de antemano durante la compilación, lo que resulta en un rendimiento rápido y una menor carga en el servidor.}
}